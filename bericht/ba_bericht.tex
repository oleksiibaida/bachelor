\documentclass[12pt, letterpaper]{article}
\usepackage[ngerman]{babel}
\usepackage{graphicx}
\usepackage{wrapfig}
\usepackage{titlesec}
\usepackage{geometry}
\usepackage[font=scriptsize]{caption}
\usepackage{blindtext}
\usepackage{hyperref}
\usepackage{tabularx}
\usepackage{subcaption}
\usepackage{verbatim}
\usepackage{fancyvrb}
\usepackage{listings}
\usepackage{xcolor}
\usepackage{multirow}
\usepackage{array} % Für die Spaltenbreitenanpassung
\usepackage{booktabs} % Für schönere Tabellen
\usepackage{geometry}
\usepackage{arydshln}
\usepackage{colortbl}
\usepackage{forest}
\usepackage{caption}
\usepackage{pifont}
\usepackage{amssymb}
\usepackage{enumitem}
\renewcommand{\lstlistlistingname}{Programmcode}

\geometry{a4paper, margin=2cm}
\hypersetup{
  colorlinks = true,
  linkcolor = black,
  urlcolor = blue,
}

% \captionsetup{justification=raggedright,singlelinecheck=false}
\geometry{
 a4paper,
 total={170mm,257mm},
 left=20mm,
 top=20mm,
 }
%  \titleformat{\section}[display]
%    {\normalfont\bfseries}{}{0pt}{\huge}

\lstdefinestyle{py}{
    language=Python,
    backgroundcolor=\color{white},
    keywordstyle=\color{blue}\bfseries,        % Default Python keywords
    commentstyle=\color{brown}\bfseries,         % Comments
    stringstyle=\color{brown},                 % Strings
    numberstyle=\tiny\color{gray},             % Line numbers
    basicstyle=\ttfamily\small,                % Base font
    identifierstyle=\color{black},             % Default for variable names
    keywordstyle=[2]\color{cyan},              % Functions
    keywordstyle=[3]\color{purple},            % Types
    breaklines=true,
    showstringspaces=false,
    tabsize=4,
    captionpos=b,
    %numbers=left,
    %numbersep=10pt,
    frame=single
}

% Define specific keywords
\lstset{
    morekeywords={import, from, def, return},  % Default Python keywords
    morekeywords=[2]{print, println, len, range, input}, % Functions
    morekeywords=[3]{int, float, str, list, dict, bool} % Types
}


\lstdefinestyle{cpp}{
    language=C,
    backgroundcolor=\color{white},
    keywordstyle=\color{blue}\bfseries,        % Standard Keywords
    commentstyle=\color{green}\itshape,        % Kommentare
    stringstyle=\color{orange},                % Strings
    numberstyle=\tiny\color{gray},             % Zeilennummern
    basicstyle=\ttfamily\small,                % Basis-Schriftart
    identifierstyle=\color{black},             % Standard für Variablennamen
    keywordstyle=[2]\color{cyan},              % Funktionen
    keywordstyle=[3]\color{purple},            % Typen
    breaklines=true,
    showstringspaces=false,
    tabsize=4,
    captionpos=b,
    %numbers=left,
    %numbersep=10pt,
    frame=single
}

% Spezifische Keywords definieren
\lstset{
    morekeywords={if, else, while, return},         % Standard C-Keywords
    morekeywords=[2]{printf, scanf, main, malloc},  % Funktionen (inkl. malloc)
    morekeywords=[3]{int, float, char, double}      % Typen
}

%\lstset{style=pycharm-light}

  
\usepackage{lipsum}  
\graphicspath{ {./Bilder/} }
\author{Oleksii Baida}
\date{Mai 2024}
\begin{document}
\begin{titlepage}
  \includegraphics[width = 0.25\pdfpagewidth]{./Bilder/FHDO.jpg}
  \begin{center}
    
    \huge \textbf{\textsf{Bachelorarbeit}} \\
    \vspace{3cm}
    \large \textbf{Oleksii Baida}\\
    \textbf{Matrikelnummer 7210384}\\
    \vspace{3cm}
    \large \textbf{Sicherheits- \& Steuerungssytem für das Haus}\\
    \vspace{1cm}
    \large \textbf{Bericht}\\
    \vspace{1cm}
    \today
  \end{center}
\end{titlepage}

\tableofcontents
\pagebreak

\section{Einleitung}

\newpage


\section{Quellen}
\begin{thebibliography}{20}
  \bibitem{pa1}
  O. Baida,
  \textit{Anbindung der Sensoren und Aktoren an den Arduino zur Realisierung eines Sicherheitssystems},
  Projektarbeit 1, 2024.

  \bibitem{code}
  O. Baida, Projektordner für PA2: \url{https://github.com/oleksiibaida/PA2.git}
  \bibitem{video}
  O. Baida, Demo-Video: \url{https://youtu.be/-Ul_ye1KLy0}
  \par \textbf{Links zur verwendeten Hardware:}
  \bibitem{arduino}
  Arduino.cc, \textit{Arduino UNO}, \url{https://docs.arduino.cc/hardware/uno-rev3/}
  \bibitem{raspi}
  Raspberry Pi Foundation, \textit{Raspberry Pi 1 B+}, \url{https://www.raspberrypi.com/products/raspberry-pi-1-model-b-plus/}
  \bibitem{esp8266}
  Espressif, \textit{ESP8266}, \url{https://www.espressif.com/}, \url{https://www.electronicwings.com/sensors-modules/esp8266-wifi-module}
  \par \textbf{Links zur verwendeten Software:}
  \bibitem{mqtt}
  Dr Andy Stanford-Clark, Arlen Nipper, \textit{Message Queuing Telemetry Transport}, \url{https://mqtt.org/}
  \bibitem{python}
  Guido van Rossum, Python Software Foundation, \textit{Python}, \url{https://www.python.org/}
  \bibitem{telegram}
  Telegram FZ-LLC, \textit{Telegram Messenger}, \url{https://github.com//telegramdesktop/tdesktop}
  \par \textbf{Linux-Packete}:
  \bibitem{hostapd}
  Jouni Malinen, \textit{hostapd}, \url{https://w1.fi/hostapd/}, Zugriff am: 19. September 2024.
  \bibitem{dnsmasq}
  Simon Kelley, \textit{dnsmasq}, \url{https://dnsmasq.org/doc.html}, Zugriff am: 20. September 2024.
  \bibitem{mosquitto}
  Eclipse Foundation, \textit{Eclipse Mosquitto}, \url{https://mosquitto.org/}
  \par \textbf{ESP- und Arduino-Bibliotheken}
  \bibitem{pubsub} 
  Knolleary, \textit{PubSubClient}, \url{https://pubsubclient.knolleary.net/}, Zugriff am: 21. Oktober 2024.
  \bibitem{espwifi}
  ESPWIFI.h, \url{https://arduino-esp8266.readthedocs.io/en/latest/esp8266wifi/readme.html}
  \bibitem{eeprom}
  EEPROM.h, \url{https://docs.arduino.cc/learn/built-in-libraries/eeprom/}
  \bibitem{keypad}
  Keypad.h \url{https://docs.arduino.cc/libraries/keypad/}
  \bibitem{scholz}
  R. Scholz, \textit{Syncloop}, Persönliche Mitteilungen
  \par \textbf{Python-Bibliotheken}
  \bibitem{paho}
  Pierre Fersing, Roger Light \textit{paho-mqtt}, \url{https://pypi.org/project/paho-mqtt/}, Zugriff am: 21. Oktober 2024.
  \bibitem{tgbot}
  Open Source, \textit{python-telegram-bot}, \url{https://docs.python-telegram-bot.org/en/v21.6/}
  \bibitem{json}
  Python Software Foundation, \textit{json}, \url{https://docs.python.org/3/library/json.html}
  \bibitem{threading}
  Python Software Foundation, \textit{threading}, \url{https://docs.python.org/3/library/threading.html}
  \bibitem{queue}
  Python Software Foundation, \textit{queue}, \url{https://docs.python.org/3/library/queue.html}
  \bibitem{sqlite}
  Gerhard Häring, \textit{sqlite3}, \url{https://docs.python.org/3/library/sqlite3.html}
  \bibitem{pyzbar}
  Lawrence Hudson, \textit{pyzbar}, \url{https://github.com/NaturalHistoryMuseum/pyzbar/}
  \bibitem{cv2}
  Intel, \textit{OpenCV}, \url{https://github.com/opencv/opencv-python}
  \bibitem{aiohttp}
  Aio-Libs, \textit{aiohttp}, \url{https://github.com/aio-libs/aiohttp}
\end{thebibliography}



\listoffigures
\addcontentsline{toc}{section}{Abbildungsverzeichnis}

\listoftables
%\addcontentsline{toc}{section}{Tabellenverzeichnis}
\lstlistoflistings
\addcontentsline{toc}{section}{Programmcode}



\end{document}
% „“